\documentclass[../DoAn.tex]{subfiles}
\begin{document}

% Chương này có độ dài không quá 10 trang. Nếu cần trình bày dài hơn, sinh viên đưa vào phần phụ lục. Chú ý đây là kiến thức đã có sẵn; SV sau khi tìm hiểu được thì phân tích và tóm tắt lại. Sinh viên không trình bày dài dòng, chi tiết.

% Với đồ án ứng dụng, sinh viên để tên chương là “Công nghệ sử dụng”. Trong chương này, sinh viên giới thiệu về các công nghệ, nền tảng sử dụng trong đồ án. Sinh viên cũng có thể trình bày thêm nền tảng lý thuyết nào đó nếu cần dùng tới.

% Với đồ án nghiên cứu, sinh viên đổi tên chương thành “Cơ sở lý thuyết”. Khi đó, nội dung cần trình bày bao gồm: Kiến thức nền tảng, cơ sở lý thuyết, các thuật toán, phương pháp nghiên cứu, v.v.

% Với từng công nghệ/nền tảng/lý thuyết được trình bày, sinh viên phải phân tích rõ công nghệ/nền tảng/lý thuyết đó dùng để để giải quyết vấn đề/yêu cầu cụ thể nào ở Chương 2. Hơn nữa, với từng vấn đề/yêu cầu, sinh viên phải liệt kê danh sách các công nghệ/hướng tiếp cận tương tự có thể dùng làm lựa chọn thay thế, rồi giải thích rõ sự lựa chọn của mình.

% Lưu ý: Nội dung ĐATN phải có tính chất liên kết, liền mạch, và nhất quán. Vì vậy, các công nghệ/thuật toán trình bày trong chương này phải khớp với nội dung giới thiệu của sinh viên ở phần trước đó.

% Trong chương này, để tăng tính khoa học và độ tin cậy, sinh viên nên chỉ rõ nguồn kiến thức mình thu thập được ở tài liệu nào, đồng thời đưa tài liệu đó vào trong danh sách tài liệu tham khảo rồi tạo các tham chiếu chéo (xem hướng dẫn ở phụ lục A.7).

\section{Ngôn ngữ C\# và framework ASP.NET Core}
\label{section:3.1}
C\# là một ngôn ngữ lập trình hướng đối tượng, đa năng, hiện đại và mạnh mẽ, được phát triển bởi Microsoft vào năm 2000 và là một phần của nền tảng .NET. C\# có thể được sử dụng để phát triển các ứng dụng web, di động và game trên nhiều nền tảng khác nhau nhờ vào các công cụ như MAUI, Unity và ASP.NET.

Một số tính năng của C\# giúp tạo ra các ứng dụng mạnh mẽ và bền bỉ hơn các ngôn ngữ hướng đối tượng Java. Như là, Garbage Collector là một công cụ tự động để quản lý bộ nhớ và thu gom các đối tượng không còn sử dụng trong ứng dụng. Language Integrated Query (LINQ) là một công cụ cho phép người lập trình truy vấn dữ liệu từ các nguồn khác nhau, như SQL Server, Oracle, XML, v.v. C\# cũng hỗ trợ các tính năng như generics, delegates, lambda expressions, extension methods, anonymous types, dynamic.

ASP.NET Core là một framework mã nguồn mở cho việc xây dựng các ứng dụng web động, API web và các dịch vụ đám mây. ASP.NET Core được xây dựng trên nền tảng .NET Core, và có thể chạy trên nhiều nền tảng khác nhau. ASP.NET Core cung cấp nhiều tính năng như dependency injection, middleware, routing, authentication, authorization, logging, testing và nhiều hơn nữa.

Thiết kế API server là một trong những lĩnh vực phổ biến của C\# và ASP.NET Core. API server là một ứng dụng web chịu trách nhiệm cho việc nhận và xử lý các yêu cầu từ các ứng dụng khác thông qua giao thức HTTP. API server có thể trả về các dữ liệu ở các định dạng khác nhau, như JSON, XML, HTML hoặc bất kỳ định dạng nào khác do người lập trình định nghĩa. API server có thể được sử dụng để cung cấp các chức năng như xác thực người dùng, quản lý phiên bản, phân quyền, bảo mật, caching, logging.
\vfill
\break


\section{Hệ cơ sở dữ liệu MongoDB}
\label{section:3.2}
MongoDB là một chương trình cơ sở dữ liệu NoSQL mã nguồn mở được thiết kế theo kiểu hướng đối tượng trong đó các bảng được cấu trúc một cách linh hoạt cho phép các dữ liệu lưu trên bảng không cần phải tuân theo một dạng cấu trúc nhất định nào. Chính do cấu trúc linh hoạt này nên MongoDB có thể được dùng để lưu trữ các dữ liệu có cấu trúc phức tạp và đa dạng và không cố định (hay còn gọi là Big Data).
\break
MongoDB sẽ lưu trữ dữ liệu dưới dạng Document JSON. Vì vậy, mỗi một collection sẽ các các kích cỡ và các document khác nhau. Bên cạnh đó, việc các dữ liệu được lưu trữ trong document kiểu JSON dẫn đến chúng được truy vấn rất nhanh. MongoDB cũng hỗ trợ các tính năng như:
\begin{itemize}
    \item Aggregation xử lý các bản ghi dữ liệu và trả về kết quả đã được tính toán. Các phép toán tập hợp nhóm các giá trị từ nhiều Document lại với nhau, và có thể thực hiện nhiều phép toán đa dạng trên dữ liệu đã được nhóm đó để trả về một kết quả duy nhất.
    \item Replication bao gồm hai hoặc nhiều bản sao của dữ liệu. Trong đó mỗi bản sao có thể đóng vai trò chính và phụ. Với cơ sở dữ liệu, nhu cầu lưu trữ lớn, đòi hỏi cơ sở dữ liệu toàn vẹn, không bị mất mát trước những sự cố ngoài ý muốn.
    \item Indexing có thể được tạo để cải thiện hiệu suất của các tìm kiếm trong MongoDB. Bất kỳ field nào trong MongoDB document đều có thể được index.
    \item Load balancing bằng cách sử dụng Sharding. Nó chạy trên nhiều máy chủ, cân bằng tải hoặc sao chép dữ liệu để giữ hệ thống luôn hoạt động trong trường hợp có lỗi về phần cứng.
\end{itemize}


\section{Docker}
\label{section:3.3}
Docker là một công cụ giúp cho việc tạo ra và triển khai các container để phát triển, chạy ứng dụng được dễ dàng. Các container là môi trường, mà ở đó lập trình viên đưa vào các thành phần cần thiết để ứng dụng của họ chạy được, bằng cách đóng gói ứng dụng cùng với container như vậy, nó đảm bảo ứng dụng chạy được và giống nhau ở các máy khác nhau (Linux, Windows, Desktop, Server ...)

Docker có vẻ rất giống máy ảo (nhiều người từng tạo máy ảo với công cụ ảo hóa như Virtual Box, VMWare), nhưng có điểm khác với VM: thay vì tạo ra toàn bộ hệ thống (dù ảo hóa), Docker lại cho phép ứng dụng sử dụng nhân của hệ điều hành đang chạy Docker để chạy ứng dụng bằng cách bổ sung thêm các thành phần còn thiếu cung cấp bởi container. Cách này làm tăng hiệu xuất và giảm kích thước ứng dụng.
\vfill
\break


\section{.NET MAUI}
\label{section:3.4}
.NET Multi-platform App UI (.NET MAUI) là một framework đa nền tảng để tạo các ứng dụng native dành cho thiết bị di động và máy tính với C\# và XAML. .NET MAUI là một phần của .NET 6 và là phiên bản tiếp theo của Xamarin.Forms. Nó cung cấp một cách để tạo ứng dụng di động và máy tính đa nền tảng với một mã nguồn duy nhất. .NET MAUI cũng hỗ trợ các tính năng như:
\begin{itemize}
    \item Có thể chạy trên nhiều nền tảng khác nhau như Android, iOS, macOS, Windows, Linux.
    \item Hỗ trợ liên kết dữ liệu (data binding) theo kiến trúc thiết kế phần mềm MVVM (model, view, model-view).
    \item Cung cấp các API hợp nhất từ các nền tảng nằm sau để truy cập các tính năng native như GPS, clipboard, pin, trạng thái mạng, ...
    \item Một dự án code duy nhất có thể chia sẻ giữa các nền tảng khác nhau và có thể sử dụng thêm các tính năng đặc biệt của từng nền tảng nếu cần.
    \item Hot reload cho phép người lập trình thay đổi code và xem kết quả ngay lập tức trên thiết bị đang chạy ứng dụng mà không cần phải build lại.
\end{itemize}

\end{document}