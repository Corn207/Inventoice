\documentclass[../DoAn.tex]{subfiles}
\begin{document}

\section{Kết luận}
% Sinh viên so sánh kết quả nghiên cứu hoặc sản phẩm của mình với các nghiên cứu hoặc sản phẩm tương tự.

% Sinh viên phân tích trong suốt quá trình thực hiện ĐATN, mình đã làm được gì, chưa làm được gì, các đóng góp nổi bật là gì, và tổng hợp những bài học kinh nghiệm rút ra nếu có.
Sau khi hoàn thiện hệ thống đề xuất, tôi nhận thấy rằng hệ thống đề xuất đã đạt được những kết quả như mong đợi. So sánh với các hệ thống hiện giờ trên thị trường, hệ thống đề xuất của tôi mang những điểm lợi như sau:
\begin{itemize}
    \item Giao diện người dùng đơn giản và tường minh, chỉ cần một ứng dụng mà có thể làm được toàn bộ chức năng quản lí. Người dùng không phải cao siêu hay là tín đồ công nghệ nhiều mà vẫn hiểu được cách thức hoạt động của hệ thống.
    \item Hệ thống có tính mở rộng cực cao, có thể được xây dựng với ứng dụng web, ứng dụng máy tính mà không phải thay đổi hệ backend nhiều. Hệ cơ sở dữ liệu không còn là vấn đề khi người dùng muốn thay đổi nền tảng.
    \item Sự bảo toàn dữ liệu, không hề có bên thứ ba nào tham gia trong quá trình cài đặt và sử dụng hệ thống nếu người dùng mong muốn.
    \item Chi phí duy trì hệ thống hoàn toàn dựa vào quy mô triển khai của người dùng.
    \item Bởi vì đối tượng sử dụng của hệ thống là các hộ kinh doanh vừa và nhỏ, hệ thống có sức chịu tải lớn và độ phản hồi rất cao.
\end{itemize}
Tuy nhiên, hệ thống vẫn còn một số hạn chế do thời gian thực hiện có hạn, cũng như khả năng của tôi còn hạn chế. Các hạn chế của hệ thống như sau:
\begin{itemize}
    \item Hệ thống chưa có những tính năng để phục hồi hay sao lưu dữ liệu.
    \item Hệ thống thiếu các tính năng thông báo về những thay đổi trong quy trình hoạt động.
    \item Sự chuyên môn hóa của hệ thống chưa cao, chưa có những tính năng đặc thù cho từng loại hình kinh doanh.
    \item Các chức năng hệ thống hiện giờ là đủ vận hành nhưng chưa có các tính năng nâng cao để tăng trải nghiệm người dùng hay tăng hiệu quả công việc.
\end{itemize}
Về mặt chức năng của hệ thống, tôi có thể nói là đã hoàn thành được những gì mà mình đã đề ra ban đầu. Tôi khá là hài lòng khi được hoàn thiện một ứng dụng có thể giúp ích cho nhiều người, cũng như là một sản phẩm thực thụ có thể sử dụng được trong thực tế. Đồng thời cũng là một trải nghiệm cá nhân có ích để tôi theo đuổi ngành công nghệ phần mềm thương mại. Tuy nhiên, vẫn còn một số điểm chưa được hoàn thiện, tôi sẽ cố gắng hoàn thiện trong tương lai. Đây cũng là cơ hội tốt để tôi ghi nhận và tích lũy sau này.
\vfill
\break

Về mặt kĩ thuật, tôi thấy mình đã học được một số kĩ năng mới sau khi hoàn thành đồ án này. Tôi đã có thể tự tạo ra một hệ thống hoàn chỉnh, từ frontend đến backend, từ cơ sở dữ liệu đến giao diện người dùng. Hơn nữa, tôi đã biết ứng dụng các nguyên lí nâng cao trong thiết kế một hệ thống thông tin vào lí thuyết mà tôi được dạy và quan sát trong thực tế. Những trải nghiệm và khó khăn khi áp dụng kiến trúc hệ thống như \textit{Clean Architecture}, \textit{Model-View-ViewModel} vào một ứng dụng thực tế cũng là một điều mà tôi rất trân trọng. Nó là một thước đo tốt cho sự ham học hỏi và tìm tòi cái mới. Cùng với đó, tôi cũng đã có thể tự tìm hiểu và áp dụng những công nghệ mới vào sản phẩm của mình. Ví dụ như lần đầu được học và sử dụng \textit{.NET MAUI} trong thiết kế ứng dụng điện thoại. Trước đây tôi chỉ biết thiết kế website, nhưng giờ đây khả năng đưa trải nghiệm người dùng tới gần hơn với ứng dụng điện thoại cũng là một điều tôi rất vui mừng.


\section{Hướng phát triển}
% Trong phần này, sinh viên trình bày định hướng công việc trong tương lai để hoàn thiện sản phẩm hoặc nghiên cứu của mình.

% Trước tiên, sinh viên trình bày các công việc cần thiết để hoàn thiện các chức năng/nhiệm vụ đã làm. Sau đó sinh viên phân tích các hướng đi mới cho phép cải thiện và nâng cấp các chức năng/nhiệm vụ đã làm.
Về định hướng tương lai của hệ thống, tôi nhận thấy rằng bài toán giúp các hộ kinh doanh vừa và nhỏ trong thời đại công nghệ 4.0 này là một bài toán thú vị và có nhiều tiềm năng. Dựa trên các phản hồi của người dùng trong giai đoạn hệ thống triển khai thử nghiệm, hệ thống sẽ cần thêm nhiều chức năng chuyên môn hóa hơn để phù hợp với nhu cầu người dùng. Thêm vào đó, hệ thống cũng cần có những tính năng mới để đạt hiệu quả hơn. Các tính năng đó có thể kể đến như:
\begin{itemize}
    \item Tính năng quản lí chi phí, giúp người dùng hiệu quả hơn việc thu chi trong quá trình kinh doanh.
    \item Tính năng quản lí sản phẩm cần thêm những chức năng cảnh báo hạn sử dụng, tái đặt hàng, thanh lí hàng tồn kho.
    \item Tính năng quản lí đơn hàng, cho phép người dùng tạo sẵn đơn hàng để có thể sử dụng lại trong tương lai.
    \item Tính năng báo cáo, tổng hợp các thông tin quan trọng, tổng quan chỉ số hỗ trợ người dùng.
    \item Cải thiện độ tương tác của ứng dụng, giúp người dùng tiết kiệm trong việc nhập liệu.
    \item Chức năng in ấn, chia sẻ hóa đơn cũng là một tính năng được chú trọng.
    \item Khả năng sao lưu và phục hồi dữ liệu, giúp chống chịu với những rủi ro có thể xảy ra.
    \item Thông báo thời gian thực, cung cấp người dùng thông tin hoạt động kinh doanh đang diễn ra như thế nào.
\end{itemize}
\vfill
\break

Đây là những tính năng mà tôi nghĩ nếu có thêm vào hệ thống sẽ giúp ích rất nhiều cho người dùng. Tuy nhiên, để có thể hoàn thiện những tính năng này, tôi cần phải có thêm nhiều kiến thức và kinh nghiệm hơn nữa. Đây cũng là một trong những động lực để tôi tiếp tục học hỏi và nghiên cứu trong tương lai.

\end{document}