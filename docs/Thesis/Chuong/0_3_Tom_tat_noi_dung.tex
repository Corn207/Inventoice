\documentclass[../DoAn.tex]{subfiles}
\begin{document}

\begin{center}
    \Large{\textbf{TÓM TẮT NỘI DUNG ĐỒ ÁN}}\\
\end{center}
\vspace{1cm}
% Sinh viên viết tóm tắt ĐATN của mình trong mục này, với 200 đến 350 từ. Theo trình tự, các nội dung tóm tắt cần có: (i) Giới thiệu vấn đề (tại sao có vấn đề đó, hiện tại được giải quyết chưa, có những hướng tiếp cận nào, các hướng này giải quyết như thế nào, hạn chế là gì), (ii) Hướng tiếp cận sinh viên lựa chọn là gì, vì sao chọn hướng đó, (iii) Tổng quan giải pháp của sinh viên theo hướng tiếp cận đã chọn, và (iv) Đóng góp chính của ĐATN là gì, kết quả đạt được sau cùng là gì. Sinh viên cần viết thành đoạn văn, không được viết ý hoặc gạch đầu dòng.
Đối với các cửa hàng vừa và nhỏ, việc vận hành và duy trì hoạt động kinh doanh là một trong những vấn đề quan trọng để cửa hàng có thể tồn tại và phát triển. Khi lưu lượng hàng hóa và khách hàng tăng lên, việc quản lý cửa hàng trở nên phức tạp hơn. Các giấy tờ chi chép thủ công không còn đáp ứng đủ nhu cầu của người quản lí nữa. Các mặt hàng trong cửa hàng không được quản lý một cách khoa học, dẫn đến việc hàng hóa bị thất thoát, hết hạn sử dụng, v.v. Và khi cửa hàng mở rộng và cần thêm nhân viên, trở ngại lớn nhất với họ là việc nhân viên mới mất bao lâu để quen việc, ghi nhớ được vị trí của hàng hóa cũng như giá cả và số lượng của chúng.

Khi thời đại công nghệ số đang phát triển, để giải quyết vấn đề hóc búa trên, các hệ thống quản lý cửa hàng cũng dần phổ biến hơn. Nhưng các giải pháp hiện tại vẫn còn nhiều hạn chế và thiếu sự đa dạng. Vì kích cỡ quy mô kinh doanh của các cá nhân quản lý cửa hàng vừa và nhỏ rất đặc thù, việc áp dụng các quy trình rườm rà phức tạp của các hệ thống đại lí lớn, cộng thêm rào cản công nghệ cũng như nhân lực để có thể triển khai các hệ thống đó là điều không thể. Vì vậy, tôi đã đề xuất một giải pháp mới, một hệ thống quản lý hóa đơn, hàng hóa, tồn kho cho các cửa hàng vừa và nhỏ, với các tính năng đơn giản, dễ sử dụng, dễ triển khai, và đặc biệt là giá thành thấp. Tận dụng từ các thiết bị thường trực như máy tính, điện thoại thông minh, việc điều hành kinh doanh trở nên dễ dàng và thuận tiện hơn bao giờ hết.

Đây là một cơ hội tốt để tôi có thể áp dụng những kiến thức đã học được trong suốt quá trình học tập và nghiên cứu. Từ việc quan sát các vấn đề thực tế và xu hướng công nghệ số mới, tôi đã thấy được sự ứng dụng của chúng trong việc phục vụ con người là vô cùng lớn. Biết bao thời gian và công sức chúng ta có thể tiết kiệm được nếu biết tận dụng những cơ hội và tiềm năng mà công nghệ số mang lại. Nguồn cảm hứng này đã thúc đẩy tôi tìm hiểu và phát triển các giải pháp công nghệ số cho các vấn đề thực tế, và tôi tin rằng đây là một xu hướng tất yếu trong tương lai.

\begin{flushright}
    Sinh viên thực hiện\\
    \begin{tabular}{@{}c@{}}
        \textit{(Ký và ghi rõ họ tên)}
    \end{tabular}
\end{flushright}

\end{document}