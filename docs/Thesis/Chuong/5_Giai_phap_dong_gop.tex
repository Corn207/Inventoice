\documentclass[../DoAn.tex]{subfiles}
\begin{document}

% Chương này có độ dài tối thiểu 5 trang, tối đa không giới hạn.\footnote{Trong trường hợp phần này dưới 5 trang thì sinh viên nên gộp vào phần kết luận, không tách ra một chương riêng rẽ nữa.} Sinh viên cần trình bày tất cả những nội dung đóng góp mà mình thấy tâm đắc nhất trong suốt quá trình làm ĐATN. Đó có thể là một loạt các vấn đề khó khăn mà sinh viên đã từng bước giải quyết được, là giải thuật cho một bài toán cụ thể, là giải pháp tổng quát cho một lớp bài toán, hoặc là mô hình/kiến trúc hữu hiệu nào đó được sinh viên thiết kế.

% Chương này \textbf{là cơ sở quan trọng} để các thầy cô đánh giá sinh viên. Vì vậy, sinh viên cần phát huy tính sáng tạo, khả năng phân tích, phản biện, lập luận, tổng quát hóa vấn đề và tập trung viết cho thật tốt.
% Mỗi giải pháp hoặc đóng góp của sinh viên cần được trình bày trong một mục độc lập bao gồm ba mục con: (i) dẫn dắt/giới thiệu về bài toán/vấn đề, (ii) giải pháp, và (iii) kết quả đạt được (nếu có).

% Sinh viên lưu ý \textbf{không trình bày lặp lại nội dung}. Những nội dung đã trình bày chi tiết trong các chương trước không được trình bày lại trong chương này. Vì vậy, với nội dung hay, mang tính đóng góp/giải pháp, sinh viên chỉ nên tóm lược/mô tả sơ bộ trong các chương trước, đồng thời tạo tham chiếu chéo tới đề mục tương ứng trong Chương 5 này. Chi tiết thông tin về đóng góp/giải pháp được trình bày trong mục đó.

% Ví dụ, trong Chương 4, sinh viên có thiết kế được kiến trúc đáng lưu ý gì đó, là sự kết hợp của các kiến trúc MVC, MVP, SOA, v.v. Khi đó, sinh viên sẽ chỉ mô tả ngắn gọn kiến trúc đó ở Chương 4, rồi thêm các câu có dạng: ``Chi tiết về kiến trúc này sẽ được trình bày trong phần 5.1". 
\section{Kiến trúc Clean Architecture}
\label{section:clean_architecture}


\subsection{Dẫn dắt vấn đề}
\label{subsection:clean_architecture-problem}
Trong quá trình thiết kế kiến trúc hệ thống, việc chia nhỏ mã nguồn và phân tách thành các tầng lớp riêng mang lại nhiều lợi ích. Mỗi tầng lớp đó sẽ mang một vài chức năng cụ thể, giúp cho việc phát triển, bảo trì trở nên dễ dàng hơn. Việc này sẽ giúp các đội lập trình viên có thể tập trung vào nhiệm vụ của họ mà không phải bận tâm đến những thay đổi của các thành viên khác. Miễn sao kiến trúc hệ thống có một quy chuẩn chung, các thành viên sẽ có thể độc lập làm việc với nhau. Ví dụ như trong kiến trúc N-tier, chúng được chia thành các tầng lớp mang nhiệm vụ cụ thể:
\begin{itemize}
    \item \textbf{Presentation Layer}: Tầng lớp này sẽ chịu trách nhiệm hiển thị giao diện người dùng, nhận các sự kiện từ người dùng và gửi đến tầng \textbf{Business Logic Layer} để xử lí các yêu cầu đó.
    \item \textbf{Business Logic Layer}: Tầng lớp này sẽ xử lý các nghiệp vụ của hệ thống. Nó sẽ nhận các yêu cầu từ tầng trên nó (\textbf{Presentation Layer}), sử dụng các tham số đó để truy cập dữ liệu ở tầng dưới (\textbf{Data Access Layer}) và cuối cùng thực hiện xử lí dữ liệu.
    \item \textbf{Data Access Layer}: Tầng lớp này chịu trách nhiệm truy cập dữ liệu từ cơ sở dữ liệu. Nó sẽ nhận các yêu cầu từ tầng (\textbf{Business Logic Layer}), truy cập vào cơ sở dữ liệu và gửi lại kết quả. Tầng này cũng định nghĩa các hàm sẵn với các câu truy vấn cơ sở dữ liệu đã được chuyên môn hóa và tối ưu sẵn, giúp cho việc sử dụng lại code được tận dụng tối đa.
    \item \textbf{Data Layer}: Tầng lớp này sẽ chịu trách nhiệm lưu trữ dữ liệu của hệ thống. Thường chúng sẽ là các hệ cơ sở dữ liệu, hay đơn thuần chỉ là các cách thức lưu trữ dữ liệu khác như file, cache, v.v.
\end{itemize}
Như đã thấy ở trên, kiến trúc hệ thống đã có sự chuyên môn hóa rõ ràng. Nhưng trong thực tế, việc thay đổi trong công nghệ sử dụng và sự thích nghi theo thị trường sẽ luôn tạo ra vấn đề. Ví dụ như khi một công nghệ mới ra đời, nó có thể giúp cho việc phát triển hệ thống trở nên dễ dàng hơn, tuy nhiên nó lại không tương thích với kiến trúc hệ thống hiện tại. Điều này dẫn đến việc phải thay đổi kiến trúc hệ thống để phù hợp với công nghệ mới.
\vfill
\break

Một ví dụ thực tiễn, khi một hệ thống khi lần đầu thiết kế, họ dùng một hệ cơ sở dữ liệu phù hợp với số lượng người dùng hiện giờ, việc thiết kế trước cho tương lai với số lượng người dùng không định hình trước được là rất tốn kém. Tuy nhiên, khi hệ thống phát triển, số lượng người dùng tăng lên, hệ cơ sở dữ liệu hiện tại không còn phù hợp nữa. Và việc thay đổi hệ cơ sở dữ liệu sẽ ảnh hưởng đến toàn bộ hệ thống. Điều này dẫn đến việc phải thay đổi toàn bộ hệ thống để phù hợp với hệ cơ sở dữ liệu mới. Điều này sẽ tốn kém thời gian và công sức của các lập trình viên. Và đây cũng là một trong những vấn đề mà kiến trúc Clean Architecture hướng đến.


\subsection{Giải pháp}
\label{subsection:clean_architecture-solution}
Để giải quyết vấn đề trên, một kiến trúc mới đã ra đời, đó là Clean Architecture. Kiến trúc này sẽ tách biệt hoàn toàn các thành phần của hệ thống với nhau, giúp cho việc thay đổi một thành phần không ảnh hưởng đến các thành phần khác. Điều này sẽ giúp cho việc phát triển và bảo trì hệ thống trở nên dễ dàng hơn. Kiến trúc này sẽ được chia thành các tầng lớp hầu như rất tương tự kiến trúc N-tier, tuy nhiên nó có một số điểm khác biệt để giải quyết vấn đề các tầng phụ thuộc nhau trong kiến trúc N-tier. Các tầng lớp của kiến trúc Clean Architecture sẽ được chia thành các tầng lớp sau:
\begin{itemize}
    \item \textbf{Presentation Layer}: Tương tự như kiến trúc N-tier.
    \item \textbf{Infrastructure Layer}: Tầng lớp này tương ứng với tầng \textbf{Data Access Layer} của kiến trúc N-tier nhưng khác ở chỗ, giờ nó phụ thuộc vào tầng \textbf{Application Layer}.
    \item \textbf{Application Layer}: Tầng này gần như tương tự với tầng \textbf{Business Logic Layer} của kiến trúc N-tier, tuy nhiên nó không phụ thuộc vào tầng \textbf{Infrastructure Layer}.
    \item \textbf{Domain Layer}: Tầng này sẽ chứa các thực thể của hệ thống, các thực thể này sẽ được định nghĩa bởi người thiết kế hệ thống. Tầng này sẽ không phụ thuộc vào bất kì tầng nào khác.
\end{itemize}
Lí do kiến trúc này thay thứ tự phụ thuộc của nhau đều có mục đích đặc biệt để giải quyết vấn đề, đặc biệt trong trường hợp này là vấn đề thay đổi hệ cơ sở dữ liệu. Với kiến trúc Clean Architecture, việc thay đổi hệ cơ sở dữ liệu sẽ không ảnh hưởng đến các tầng khác. Vì các tầng khác không phụ thuộc vào tầng \textbf{Infrastructure Layer} nữa. Đây là một trong các lợi ích của kiến trúc này.

Cụ thể, đặt ra một ví dụ giả thiết, ta đang thiết kế hệ thống với kiến trúc N-tier với hệ cơ sở dữ liệu SQL Server. Bởi vì tính chất của hệ thống giả dụ là thay đổi cấu trúc dữ liệu thường xuyên như là mạng xã hội,... khiến các thực thể biểu diễn cho các bảng trong SQL Server thường xuyên phải cấu trúc lại schema rất nhiều. Điều này khiến việc duy trì và phát triển gặp nhiều bất tiện và tốn kém thời gian. Vì vậy, ta sẽ quyết định dùng hệ cơ sở dữ liệu MongoDB để thay thế vì đặc tính của nó là document database, có thể thay đổi cấu trúc dữ liệu một cách dễ dàng. Tuy nhiên, với kiến trúc N-tier, việc thay đổi hệ cơ sở dữ liệu sẽ ảnh hưởng đến tầng \textbf{Data Access Layer} và suy ra, các tầng bên trên như \textbf{Business Logic Layer}, v.v. cũng phải thay đổi theo. Nếu ta biết trước được điều này, cách tốt nhất là phải có một kiến trúc có khả năng chống chịu được thay đổi này.

Chi tiết lựa chọn làm tầng \textbf{Infrastructure Layer} phụ thuộc ngược lại tầng \textbf{Application Layer} sẽ được trình bày ở như sau. Sự thay đổi trong mã nguồn trong tầng nằm trên sẽ không ảnh hưởng đến tầng nằm dưới. Vì vậy, tầng \textbf{Application Layer} sẽ phụ thuộc vào tầng \textbf{Infrastructure Layer}. Nhưng làm sao để tầng \textbf{Application Layer} có thể gọi các hàm truy vấn cơ sở dữ liệu mà không phụ thuộc vào tầng \textbf{Infrastructure Layer} được. Ở đây chúng ta sẽ áp dụng một kĩ thuật trong lập trình hướng đối tượng (Object Oriented Programming) để giải quyết vấn đề này.

Thay vì chúng ta định nghĩa đầy đủ logic gọi truy vấn cơ sở dữ liệu trong một tầng. Chúng ta sẽ chỉ định nghĩa đầu vào và đầu ra của các hàm truy vấn cơ sở dữ liệu. OOP cung cấp khái niệm Interface cho việc định nghĩa hàm này. Ví dụ, việc truy vấn sản phẩm cần đầu vào là một biến kiểu chữ Id và đầu ra là một biến kiểu Product. Ta chỉ việc viết một interface có tên là IProductRepository với một hàm có tên là GetProductById() trong chính tầng \textbf{Application Layer} hoặc các tầng khác ngang hàng hoặc thấp hơn. Còn khi ta định nghĩa cách gọi truy vấn cơ sở dữ liệu, ta sẽ viết một class có tên là ProductRepository kế thừa từ interface IProductRepository và định nghĩa hàm GetProductById() trong tầng \textbf{Infrastructure Layer}. Và tầng \textbf{Application Layer} sẽ gọi hàm GetProductById() thông qua interface IProductRepository. Điều này sẽ giúp cho việc thay đổi hệ cơ sở dữ liệu không ảnh hưởng đến tầng \textbf{Application Layer}.

Sau khi đã thiết kế xong được 2 tầng \textbf{Application Layer} và \textbf{Infrastructure Layer}. Ở mã nguồn sản phẩm cuối, người lập trình chỉ cần nối các tầng với nhau và có thể chọn tầng \textbf{Infrastructure Layer} phù hợp với hệ cơ sở dữ liệu mà họ muốn. Một kĩ thuật trong lập trình hướng đối tượng khác để thực hiện sự lựa chọn này là Dependency Injection. Chúng mang vai trò như một trung tâm khởi tạo và duy trì vòng đời của các lớp trong hệ thống, ngoài ra chúng còn cho phép người lập trình có thể lựa chọn các lớp nào sẽ là lớp được sử dụng cho interface nào. Dẫn tới kĩ thuật này sẽ đưa vào các lớp của tầng \textbf{Application Layer} một đối tượng của tầng \textbf{Infrastructure Layer} thông qua interface. Việc chọn hệ cơ sở dữ liệu dễ dàng chỉ bằng thay đổi một dòng code đơn giản.
\vfill
\break

\subsection{Kết quả đạt được}
\label{subsection:clean_architecture-result}
Áp dụng các kĩ thuật được mô tả trong phần \ref{subsection:clean_architecture-solution}, hệ thống của chúng ta có thể đáp ứng được sự biến đối và tiến hóa của công nghệ. Không những thế, nó còn đem lại khả năng mở rộng, khả năng thích ứng với mọi môi trường. Một hệ thống nay có thể cho phép người quản lý tha hồ lựa chọn hệ cơ sở dữ liệu mà họ muốn mà không ảnh hưởng đến các thành phần khác của hệ thống.

Ví dụ vì hệ thống theo kiến trúc \textit{Clean Architecture} nên ta có thể tạo ra hai biến thể của tầng \textbf{Infrastructure Layer} là \textbf{SQLServerInfrastructureLayer} và \textbf{MongoDBInfrastructureLayer}. Và hệ thống triển khai đã có sẵn SQL Server, họ chỉ cần nối tầng \textbf{SQLServerInfrastructureLayer} vào tầng \textbf{Application Layer} thông qua \textbf{Dependency Injection}. Và khi họ muốn chuyển sang MongoDB, họ chỉ cần thay đổi một dòng code trong \textbf{Dependency Injection} để nối tầng \textbf{MongoDBInfrastructureLayer} vào tầng \textbf{Application Layer}.

Những ví dụ trên đã được thấy trong thực tế rất nhiều khi Discord chuyển từ PostgreSQL sang Cassandra, khi Facebook chuyển từ MySQL sang Cassandra, v.v. Điều này cho thấy kiến trúc \textit{Clean Architecture} là một kiến trúc phù hợp với thực tế và có thể giải quyết được nhiều vấn đề và thay đổi trong hệ cơ sở dữ liệu là một ví dụ điển hình.

Khi sau một thời gian thực hành và nghiên cứu, và được biết đến những vấn đề trên đã thúc đẩy tôi tìm hiểu kĩ hơn về các kiến trúc hệ thống. Chúng khiến tôi thấy thú vị và thấy được một trong các ứng dụng của lập trình hướng đối tượng tạo nên sự linh hoạt và dễ dàng trong việc phát triển hệ thống. Và đồ án tốt nghiệp này là một cơ hội tốt để tôi có thể áp dụng những kiến thức đã học vào thực tế.
\vfill
\break


\section{Kiến trúc Model-View-ViewModel}
\label{section:mvvm}


\subsection{Dẫn dắt vấn đề}
\label{subsection:mvvm-problem}
Các kĩ sư phần mềm thường sử dụng nhiều mô hình, công cụ khác nhau để hỗ trợ cho việc phát triển và kiểm thử ứng dụng và website. Việc phân tách ứng dụng thành các thành phần riêng biệt giúp cho đội ngũ phát triển có thể độc lập làm việc với nhau. Một trong những mô hình được sử dụng phổ biến nhất là mô hình Model-View-Controller (MVC). Mô hình này phân tách ứng dụng thành 3 thành phần chính:
\begin{itemize}
    \item \textbf{Model}: Thành phần này chịu trách nhiệm cho việc lưu trữ dữ liệu và xử lý dữ liệu.
    \item \textbf{View}: Thành phần này chứa các giao diện người dùng và hiển thị dữ liệu.
    \item \textbf{Controller}: Thành phần này chịu trách nhiệm điều hướng các yêu cầu từ người dùng từ \textbf{View} tới \textbf{Model}.
\end{itemize}
Nhưng với sự phát triển của công nghệ, các ứng dụng ngày càng phức tạp hơn. Điểm bất cập của mô hình này là logic xử lí UI và xử lí dữ liệu từ \textbf{Model} thường đi cùng với nhau khiến cho việc phát triển và bảo trì trở nên khó khăn hơn. Hơn nữa, với xu hướng giao diện ứng dụng ngày càng hiện đại, sẽ rất khó để áp dụng các kĩ thuật xác thực đầu vào người dùng. Hơn nữa, mô hình này gây cho việc sử dụng lại code trở nên khó khăn và quy trình kiểm thử cũng rất rắc rối. Mô hình này được sử dụng rộng rãi trong các ứng dụng web. Tuy nhiên, nó lại không phát huy được điểm mạnh của các ứng dụng di động.


\subsection{Giải pháp}
\label{subsection:mvvm-solution}
Vì vậy, một mô hình khác được tạo ra để phù hợp với các ứng dụng di động, đó là mô hình Model-View-ViewModel (MVVM). Mô hình này được sử dụng rộng rãi trong các ứng dụng di động và ứng dụng web đơn trang (Single Page Application - SPA). Mô hình này cũng phân tách ứng dụng thành 3 thành phần chính:
\begin{itemize}
    \item \textbf{Model}: Thành phần này chịu trách nhiệm cho việc lưu trữ dữ liệu và xử lý dữ liệu.
    \item \textbf{View}: Thành phần này chứa các giao diện người dùng và hiển thị dữ liệu.
    \item \textbf{ViewModel}: Thành phần này đóng vai trò trung gian giữa \textbf{View} và \textbf{Model}. Nó sẽ xử lí các hoạt động truyền tải dữ liệu giữa \textbf{View} và \textbf{Model}.
\end{itemize}
\vfill
\break

Điều khác biệt của mô hình này so với mô hình MVC là ở \textbf{ViewModel}. Nó không hề biết đến, thậm chí cần đến sự tồn tại của \textbf{View}, điều này có thể tách được logic xử lí dữ liệu của giao diện người dùng với thiết kế giao diện người dùng. Sự phân hóa này có thể giúp cho người thiết kế và người lập trình được tự do hơn. Cách \textbf{ViewModel} truyền tải dữ liệu không còn đơn thuần là lắng nghe thay đổi từ \textbf{View} và bảo \textbf{Model} phải thay đổi theo nữa. Mà nó sẽ thông qua các \textbf{Binding}. \textbf{Binding} được hiểu như là một đối tượng đặc biệt trong việc chuyển hóa từ kiểu dữ liệu này thành kiểu dữ liệu khác. Lợi dụng điều này, mô hình tạo lên một cơ chế tự động thay đổi dữ liệu gọi là \textbf{Two-way Data Binding}. Cơ chế này sẽ tự động cập nhật dữ liệu từ \textbf{View} tới \textbf{ViewModel} và ngược lại. Khi đó \textbf{ViewModel} như là một biến thể dữ liệu của \textbf{Model} dành riêng để trình diễn.

Ban đầu mô hình này được thiết kế cho các ứng dụng WPF (Windows Presentation Foundation) của Microsoft cho máy tính Windows. Tuy nhiên bởi vì lợi ích của nó, rất nhiều nền tảng, framework khác đã ứng dụng vào trong các ứng dụng của họ. Dần dần, nó đã trở thành một mô hình phổ biến trong các ứng dụng di động và ứng dụng web đơn trang.


\subsection{Kết quả đạt được}
\label{subsection:mvvm-result}
Khi lựa chọn công nghệ để thiết kế ứng dụng điện thoại, đơn thuần ban đầu tôi chỉ muốn tiếp tục sử dụng hệ sinh thái .NET để tăng khả năng sử dụng lại code. Và tôi đã tìm hiểu về .NET MAUI, một framework cho phép viết ứng dụng di động, máy tính bằng ngôn ngữ C\#. Tuy nhiên, khi tôi tìm hiểu sâu hơn về framework này, rất nhiều tài liệu đều hướng đến sử dụng mô hình MVVM. Từ đây, tôi được biết thêm một mô hình hiện đại mà tôi chưa từng biết đến ngoài mô hình quá đỗi nổi tiếng MVC. Tôi đã tìm hiểu sâu hơn về mô hình này và thấy được sự linh hoạt của nó. Đây là một sự tình cờ tuyệt vời và cũng là một trong những điều mới tôi học được trong quá trình làm đồ án tốt nghiệp này.

\end{document}